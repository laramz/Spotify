% Options for packages loaded elsewhere
\PassOptionsToPackage{unicode}{hyperref}
\PassOptionsToPackage{hyphens}{url}
\PassOptionsToPackage{dvipsnames,svgnames*,x11names*}{xcolor}
%
\documentclass[
  11pt,
]{book}
\usepackage{amsmath,amssymb}
\usepackage[]{mathpazo}
\usepackage{ifxetex,ifluatex}
\ifnum 0\ifxetex 1\fi\ifluatex 1\fi=0 % if pdftex
  \usepackage[T1]{fontenc}
  \usepackage[utf8]{inputenc}
  \usepackage{textcomp} % provide euro and other symbols
\else % if luatex or xetex
  \usepackage{unicode-math}
  \defaultfontfeatures{Scale=MatchLowercase}
  \defaultfontfeatures[\rmfamily]{Ligatures=TeX,Scale=1}
\fi
% Use upquote if available, for straight quotes in verbatim environments
\IfFileExists{upquote.sty}{\usepackage{upquote}}{}
\IfFileExists{microtype.sty}{% use microtype if available
  \usepackage[]{microtype}
  \UseMicrotypeSet[protrusion]{basicmath} % disable protrusion for tt fonts
}{}
\makeatletter
\@ifundefined{KOMAClassName}{% if non-KOMA class
  \IfFileExists{parskip.sty}{%
    \usepackage{parskip}
  }{% else
    \setlength{\parindent}{0pt}
    \setlength{\parskip}{6pt plus 2pt minus 1pt}}
}{% if KOMA class
  \KOMAoptions{parskip=half}}
\makeatother
\usepackage{xcolor}
\IfFileExists{xurl.sty}{\usepackage{xurl}}{} % add URL line breaks if available
\IfFileExists{bookmark.sty}{\usepackage{bookmark}}{\usepackage{hyperref}}
\hypersetup{
  pdftitle={Data Science Project Guide: Spotify},
  pdfauthor={TechAcademy e.V.},
  colorlinks=true,
  linkcolor=Maroon,
  filecolor=Maroon,
  citecolor=Blue,
  urlcolor=blue,
  pdfcreator={LaTeX via pandoc}}
\urlstyle{same} % disable monospaced font for URLs
\usepackage[top=0.5in, bottom=1.5in, left=1in, right=1in, a4paper]{geometry}
\usepackage{longtable,booktabs,array}
\usepackage{calc} % for calculating minipage widths
% Correct order of tables after \paragraph or \subparagraph
\usepackage{etoolbox}
\makeatletter
\patchcmd\longtable{\par}{\if@noskipsec\mbox{}\fi\par}{}{}
\makeatother
% Allow footnotes in longtable head/foot
\IfFileExists{footnotehyper.sty}{\usepackage{footnotehyper}}{\usepackage{footnote}}
\makesavenoteenv{longtable}
\usepackage{graphicx}
\makeatletter
\def\maxwidth{\ifdim\Gin@nat@width>\linewidth\linewidth\else\Gin@nat@width\fi}
\def\maxheight{\ifdim\Gin@nat@height>\textheight\textheight\else\Gin@nat@height\fi}
\makeatother
% Scale images if necessary, so that they will not overflow the page
% margins by default, and it is still possible to overwrite the defaults
% using explicit options in \includegraphics[width, height, ...]{}
\setkeys{Gin}{width=\maxwidth,height=\maxheight,keepaspectratio}
% Set default figure placement to htbp
\makeatletter
\def\fps@figure{htbp}
\makeatother
\setlength{\emergencystretch}{3em} % prevent overfull lines
\providecommand{\tightlist}{%
  \setlength{\itemsep}{0pt}\setlength{\parskip}{0pt}}
\setcounter{secnumdepth}{5}
\newcommand{\sectionbreak}{\clearpage}
\usepackage{booktabs}
\RequirePackage{fix-cm}
\usepackage[many]{tcolorbox}
\usepackage{xcolor}

\definecolor{r}{HTML}{2369bd}
\definecolor{p}{HTML}{ffde4d}
\definecolor{boxgrey}{HTML}{fefefe}

\newtcolorbox{rbox}{
  colback=boxgrey,
  colframe=r,
  coltext=black,
  boxsep=5pt,
  arc=4pt}
  
\newtcolorbox{pbox}{
  colback=boxgrey,
  colframe=p,
  coltext=black,
  boxsep=5pt,
  arc=4pt}


\newenvironment{tips}[1]
  {
  \begin{itemize}
  \footnotesize
  \renewcommand{\labelitemi}{
    \raisebox{-.7\height}[0pt][0pt]{
      {\setkeys{Gin}{width=3em,keepaspectratio}
        \includegraphics{images/#1.png}}
    }
  }
  \setlength{\fboxsep}{1em}
  \begin{rbox}
  \item
  }
  {
  \end{rbox}
  \end{itemize}
  }
  
  
\newenvironment{tipsp}[1]
  {
  \begin{itemize}
  \footnotesize
  \renewcommand{\labelitemi}{
    \raisebox{-.7\height}[0pt][0pt]{
      {\setkeys{Gin}{width=3em,keepaspectratio}
        \includegraphics{images/#1.png}}
    }
  }
  \setlength{\fboxsep}{1em}
  \begin{pbox}
  \item
  }
  {
  \end{pbox}
  \end{itemize}
  }
  
\usepackage{fancyhdr}
\pagestyle{fancyplain}
\usepackage{setspace}
\usepackage{chngcntr}
\onehalfspacing

\usepackage{titling}
\pretitle{\begin{center}\LARGE\includegraphics[width=6cm]{plots/TA_Logo.png}\\[\bigskipamount]}
\posttitle{\end{center}}
\ifluatex
  \usepackage{selnolig}  % disable illegal ligatures
\fi
\usepackage[]{natbib}
\bibliographystyle{apalike}

\title{Data Science Project Guide: Spotify}
\author{TechAcademy e.V.}
\date{Summer Term 2021}

\begin{document}
\maketitle

\clearpage

\addtolength{\headheight}{17.82275pt}
\rhead{\includegraphics[height=0.5cm]{plots/TA_logo.png}}

\fancyfoot{}
\fancyfoot[R]{\thepage}
\addtolength{\headheight}{17.82275pt}

\fancyfoot[L]{Data Science Project Guide | Spotify | \copyright\ 2021, TechAcademy e.V.}

\renewcommand{\headrulewidth}{0.25pt}
\renewcommand{\footrulewidth}{0.25pt}

\tableofcontents
\clearpage

\hypertarget{welcome}{%
\chapter{Welcome!}\label{welcome}}

In the first few chapters you will be introduced to the basics of the \texttt{R} and \texttt{Python} tracks respectively and you will find helpful explanations to questions you might have in the beginning of your coding journey. There will be a quick introduction to the Data Science track so that you can get started with the project quickly. So let's get started with the basics!

In all tracks you will work on your project in small groups of fellow students. This not only helps you get the project done faster, it also helps make your results even better. Our experience shows: The different backgrounds of the members and discussing different opinions and ideas will produce the best results. Besides, it is of course more fun to work on a project together than to code alone!\\
The groups can consist of a maximum of three members. You can choose your two teammates independently, we won't interfere with your arrangements. It is important that all group members complete the same level of difficulty (beginner or advanced), since the tasks are different in scope for each level. We explicitly encourage you to collaborate with students from different departments. This not only allows you to get to know people from other departments, but may even give you a whole new perspective on the project and tasks.
When submitting it is important to note: for a certificate, each person must submit the project individually. However, this can be identical within your group. You can get more information at our first Coding Meetup on \textbf{December 9, 2020}.

This Airbnb case study and the associated project guide was developed and written entirely from scratch by TechAcademy's Data Science team. Lara Zaremba, Lukas Jürgensmeier, and Annalisa Strauß developed the project in \texttt{R}, while Felix Schneider and Manuel Mair am Tinkhof developed it in \texttt{Python}. We would also like to give special thanks to Benjamin Lucht for his extensive input when conceiving this project.

\hypertarget{whats-data-science-and-how-do-i-do-it}{%
\chapter{What's Data Science and How Do I Do It?}\label{whats-data-science-and-how-do-i-do-it}}

Data science is a multi-layered field in which the use of the latest machine learning methods is only a sub-area. To get there, you'll need many steps before -- from collecting to manipulating to exploring the data. And eventually, you will need to somehow communicate your findings.

But first things first. To analyze data, it must first be obtained. You need to know where to obtain it and how to integrate it in your respective tools. The data is rarely available as it would be needed for further processing. Familiarizing yourself with the information available, cleaning it up and processing it into the desired formats that can be read by humans and machines are important steps that often make up a large part of the work.

Before the obtained data can be analyzed, the right tool must be selected and mastered: the programming language. The most often used languages for Data Science are \texttt{R}, which was explicitly developed for statistics, and \texttt{Python}, which is characterized by its additional versatility. The data scientist does not have to be a perfect software developer who masters every detail and paradigm, but the competent handling of syntax and idiosyncrasies is essential.
There are some well-developed method collections, so-called packages or libraries, which provide a lot of functionality. The use of these collections also has to be learned and mastered.
Once all of this is achieved, the data can finally be analyzed. Here too, it is important to know and understand the multitude of statistical approaches in order to be able to choose the right method for the problem at hand. The newest, best, most beautiful neural network is not always the solution for everything.\\
One step is still missing in the data science process: understanding and communicating the results. The results are often not spontaneously intuitive or sometimes even surprising. Here, the specific expertise and creativity can be played out, especially in the visualization.

\hypertarget{whats-r}{%
\section{What's R?}\label{whats-r}}

\texttt{R} is a programming language that was developed by statisticians in the early 1990s for use in the calculation and visualization of statistical applications. A lot has happened since then and by now, \texttt{R} is one of the most widely used programming languages in the field of data science. Code in \texttt{R} does not have to be compiled, but can be used interactively and dynamically. This makes it possible to quickly gain basic knowledge about existing data and to display it graphically.\\
\texttt{R} offers much more than just programming, but also a complete system for solving statistical problems. A large number of packages and interfaces are available, with which the functionality can be expanded and integration into other applications is made possible.

\hypertarget{rstudio-cloud}{%
\subsection{RStudio Cloud}\label{rstudio-cloud}}

Before you can use \texttt{R}, you usually have to install some separate programs locally on your computer. Typically, you first install a ``raw'' version of \texttt{R}. In theory, you can then already start programming. However, it is very difficult to carry out an entire project with it. That's why there is RStudio, an Integrated Development Environment (IDE) for \texttt{R}. This includes many essential features that simplify programming with \texttt{R}. Among other things, an auto-completion of your code, a nicely structured user interface and many expansion options.

Experience has shown that installing \texttt{R} and RStudio locally takes some effort. Fortunately, RStudio also has a cloud solution that eliminates these steps: RStudio Cloud. There it is possible to edit your project in exactly the same IDE in the browser without any prior installations. You can also easily switch your project from private to public and give your team an insight into your code via a link or by giving them access to the workspace directly. In this way you are able to easily exchange ideas within your team.

We will introduce RStudio Cloud and unlock access to our workspace on our first Coding Meetup. Until then, focus on learning the ``hard skills'' of programming with your courses on DataCamp. This brings us to your curriculum in the next section.

\hypertarget{curriculum}{%
\subsection{Curriculum}\label{curriculum}}

The following list shows the required DataCamp courses for the Data Science with \texttt{R} Track. As a beginner, please stick to the courses of the ``beginner'' program, ambitious beginners can of course also take the advanced courses afterwards. However, the courses should be worked through in the order in which they are listed.\\
The same applies to the advanced courses. Here, too, the specified courses should be processed in the given order. Since it can of course happen that you have already mastered the topics of an advanced course, individual courses can be replaced. The topics of the advanced courses are given in key points. If these key points seem familiar to you, then take a look at the table of contents of the corresponding DataCamp course. If you are convinced that this course does not provide any added value for you, it can be replaced by one of the courses in the ``Exchange Pool'' (see list). However, this exchange course should not be processed until all other courses in the advanced course have been completed.

Both beginners and advanced learners must have completed at least two thirds of the curriculum in order to receive the certificate. For the beginners this means at least up to the course ``Data Visualization with ggplot2 (Part 1)'' and for the advanced at least up to ``Supervised Learning in \texttt{R}: Classification''. In addition, at least two thirds of the project tasks must have been completed.

\begin{tips}r

\textbf{R Fundamentals (Beginner)}

\begin{itemize}
\tightlist
\item
  \href{https://www.datacamp.com/courses/free-introduction-to-r}{Introduction to R (4h)}
\item
  \href{https://www.datacamp.com/courses/intermediate-r}{Intermediate R (6h)}
\item
  \href{https://www.datacamp.com/courses/importing-data-in-r-part-1}{Introduction to Importing Data in R (3h)}
\item
  \href{https://www.datacamp.com/courses/cleaning-data-in-r}{Cleaning Data in R (4h)}
\item
  \href{https://www.datacamp.com/courses/data-manipulation-with-dplyr-in-r}{Data Manipulation with dplyr (4h)}
\item
  \href{https://www.datacamp.com/courses/data-visualization-with-ggplot2-1}{Data Visualization with ggplot2 (Part1) (5h)}
\item
  \href{https://www.datacamp.com/courses/exploratory-data-analysis}{Exploratory Data Analysis in R (4h)}
\item
  \href{https://www.datacamp.com/courses/correlation-and-regression}{Correlation and Regression in R (4h)}
\item
  \href{https://www.datacamp.com/courses/multiple-and-logistic-regression}{Multiple and Logistic Regression in R (4h)}
\end{itemize}

\textbf{Machine Learning Fundamentals in R (Advanced)}

\begin{itemize}
\tightlist
\item
  \href{https://www.datacamp.com/courses/intermediate-r}{Intermediate R (6h)}: conditionals, loops, functions, apply
\item
  \href{https://www.datacamp.com/courses/importing-data-in-r-part-1}{Introduction to Importing Data in R (3h)}: utils, readr, data.table, XLConnect
\item
  \href{https://www.datacamp.com/courses/cleaning-data-in-r}{Cleaning Data in R (4h)}: raw data, tidying \& preparing data
\item
  \href{https://www.datacamp.com/courses/importing-cleaning-data-in-r-case-studies}{Importing \& Cleaning Data in R: Case Studies (4h)}: case studies
\item
  \href{https://www.datacamp.com/courses/data-visualization-with-ggplot2-1}{Data Visualization with ggplot2 (Part1) (5h)}: aesthetics, geometries, qplot
\item
  \href{https://www.datacamp.com/courses/supervised-learning-in-r-classification}{Supervised Learning in R: Classification (4h)}: kNN, naive bayes, logistic regression, classification trees
\item
  \href{https://www.datacamp.com/courses/supervised-learning-in-r-regression}{Supervised learning in R: Regression (4h)}: linear \& non-linear regression, tree-based methods
\item
  \href{https://www.datacamp.com/courses/unsupervised-learning-in-r}{Unsupervised Learning in R (4h)}: k-means, clustering, dimensionality reduction
\item
  \href{https://www.datacamp.com/courses/machine-learning-toolbox}{Machine Learning with caret in R (4h)}: train()-function, cross-validation, auc
\end{itemize}

\textbf{Data Science R (Advanced) -- Exchange Pool}

\begin{itemize}
\tightlist
\item
  \href{https://www.datacamp.com/courses/data-visualization-with-ggplot2-2}{Data Visualization with ggplot2 (Part 2) (5h)}
\item
  \href{https://www.datacamp.com/courses/interactive-maps-with-leaflet-in-r}{Interactive Maps with leaflet in R (4h)}
\item
  \href{https://www.datacamp.com/courses/machine-learning-in-the-tidyverse}{Machine Learning in Tidyverse (5h)}
\item
  \href{https://www.datacamp.com/courses/writing-efficient-r-code}{Writing Efficient R Code (4h)}
\item
  \href{https://www.datacamp.com/courses/support-vector-machines-in-r}{Support Vector Machines in R (4h)}
\item
  \href{https://www.datacamp.com/courses/supervised-learning-in-r-case-studies}{Supervised Learning in R: Case Studies (4h)}
\item
  \href{https://www.datacamp.com/courses/optimizing-r-code-with-rcpp}{Optimizing R Code with Rcpp (4h)}
\end{itemize}

\end{tips}

\hypertarget{links}{%
\subsection{Links}\label{links}}

\begin{itemize}
\tightlist
\item
  RStudio Cheat Sheets: \url{https://rstudio.cloud/learn/cheat-sheets}
\item
  RMarkdown Explanation (to document your analyses): \url{https://rmarkdown.rstudio.com/lesson-1.html}
\item
  StackOverflow (forum for all kinds of coding questions): \url{https://stackoverflow.com/}
\item
  CrossValidated (Statistics and Data Science forum): \url{https://stats.stackexchange.com/}
\end{itemize}

\hypertarget{whats-python}{%
\section{What's Python?}\label{whats-python}}

\texttt{Python} is a dynamic programming language. The code is executed in the interpreter, which means that the code does not first have to be compiled. This makes \texttt{Python} very easy and quick to use. The good usability, easy readability and simple structuring were and still are core ideas in the development of this programming language.
Basically, you can use \texttt{Python} to program according to any paradigm, whereby structured and object-oriented programming is easiest due to the structure of the language, but functional or aspect-oriented programming is also possible. These options give users great freedom to design projects the way they want, but also great freedom to write code that is difficult to understand and confusing. For this reason, certain standards that are based on the so-called \texttt{Python} Enhancement Proposals (PEP) have developed over the decades.

\hypertarget{anaconda-and-jupyter}{%
\subsection{Anaconda and Jupyter}\label{anaconda-and-jupyter}}

Before you can use \texttt{Python}, it must be installed on the computer. \texttt{Python} is already installed on Linux and Unix systems (such as macOS), but often it is an older version. Since there are differences in the handling of \texttt{Python} version 2 -- which is not longer supported anymore -- and version 3, we decided to work with version 3.6 or higher.

One of the easiest ways to get both \texttt{Python} and most of the best known programming libraries is to install Anaconda. There are detailed explanations for the installation on all operating systems on the \href{https://docs.anaconda.com/anaconda/install/}{website} of the provider.

With Anaconda installed, all you have to do is open the Anaconda Navigator and you're ready to go. There are two ways to get started: Spyder or Jupyter. Spyder is the integrated development environment (IDE) for \texttt{Python} and offers all possibilities from syntax highlighting to debugging (links to tutorials below). The other option is to use Jupyter or Jupyter notebooks. It is an internet technology based interface for executing commands. The big advantage of this is that you can quickly write short code pieces and try them out interactively without writing an entire executable program.
Now you can get started! If you have not worked with Jupyter before, we recommend that you complete the course on DataCamp (\url{https://www.datacamp.com/projects/33}) first. There you will get to know many tips and tricks that will make your workflow with Jupyter much easier.

In order to make your work and, above all, the collaboration easier, we are working with the Google Colab platform that contains a Jupyter environment with the necessary libraries. You can then import all the data necessary for the project with Google Drive. We will introduce this environment during our first Coding Meetup. Until then, focus on learning the ``hard skills'' of programming with your courses on DataCamp. This brings us to your curriculum in the next section.

\hypertarget{curriculum-1}{%
\subsection{Curriculum}\label{curriculum-1}}

The following list shows the DataCamp courses for the \texttt{Python} data science track. As a beginner, please follow the courses for the beginner level. These should be processed in the order in which they are listed.\\
The same applies to the advanced courses. Here, too, the specified courses should be processed in the given order. Since it can of course happen that you have already mastered the topics of an advanced course, individual courses can be replaced. The topics of the advanced courses are given in brief. If these key points seem familiar to you, then take a look at the table of contents of the corresponding DataCamp course.
If you are convinced that this course does not provide any added value for you, it can be replaced by one of the courses in the ``Exchange Pool'' (see list). However, this course should not be processed until all other courses in the intermediate \texttt{Python} course have been completed.

Both beginners and advanced learners must have completed at least two thirds of the curriculum in order to receive the certificate. For beginners this means at least up to the course \href{https://learn.datacamp.com/courses/merging-dataframes-with-pandas}{Merging DataFrames with pandas (4h)} (including this course) and for advanced learners at least up to the \href{https://learn.datacamp.com/courses/exploratory-data-analysis-in-python}{Exploratory Data Analysis in Phyton (4h)} (including this course). In addition, at least two thirds of the project tasks have to be completed.

\begin{tipsp}p

\textbf{Python Fundamentals (Beginner) }

\begin{itemize}
\tightlist
\item
  \href{https://www.datacamp.com/courses/introduction-to-data-science-in-python}{Introduction to Data Science in Python (4h)}
\item
  \href{https://www.datacamp.com/courses/intermediate-python-for-data-science}{Intermediate Python (4h)}
\item
  \href{https://www.datacamp.com/courses/python-data-science-toolbox-part-1}{Python for Data Science Toolbox (Part 1) (3h)}
\item
  \href{https://www.datacamp.com/courses/introduction-to-matplotlib}{Introduction to Data Visualization with Matplotlib (4h)}
\item
  \href{https://www.datacamp.com/courses/manipulating-dataframes-with-pandas}{Manipulating DataFrames with pandas (4h)}
\item
  \href{https://www.datacamp.com/courses/merging-dataframes-with-pandas}{Merging DataFrames with pandas (4h)}
\item
  \href{https://www.datacamp.com/courses/exploratory-data-analysis-in-python}{Exploratory Data Analysis in Phyton (4h)}
\item
  \href{https://www.datacamp.com/projects/33}{Introduction to DataCamp Projects (2h)}
\item
  \href{https://www.datacamp.com/courses/introduction-to-linear-modeling-in-python}{Introduction to Linear Modeling in Python (4h)}
\end{itemize}

\textbf{Data Science with Python (Advanced) }

\begin{itemize}
\tightlist
\item
  \href{https://www.datacamp.com/courses/intermediate-python-for-data-science}{Intermediate Python (4h)}: Matplotlib, Dict, Pandas, Loops
\item
  \href{https://www.datacamp.com/courses/python-data-science-toolbox-part-1}{Python Data Science Toolbox (Part 1) (3h)}: Default arguments, Lambdas, Error handling
\item
  \href{https://www.datacamp.com/courses/python-data-science-toolbox-part-2}{Python Data Science Toolbox (Part 2) (4h)}: Iterators, generators, List comprehension
\item
  \href{https://www.datacamp.com/courses/cleaning-data-in-python}{Cleaning Data in Python (4h)}: Using pandas for Data cleaning
\item
  \href{https://www.datacamp.com/projects/82}{Exploring the Bitcoin Cryptocurrency Market (3h)}: Small project
\item
  \href{https://www.datacamp.com/courses/exploratory-data-analysis-in-python}{Exploratory Data Analysis in Phyton (4h)}: How to start a data analysis
\item
  \href{https://www.datacamp.com/courses/introduction-to-linear-modeling-in-python}{Introduction to Linear Modeling in Python (4h)}: Linear Regression, sklearn
\item
  \href{https://www.datacamp.com/courses/supervised-learning-with-scikit-learn}{Supervised Learning with Scikit-Learn (4h)}: Classification, Regression, Tuning
\item
  \href{https://www.datacamp.com/courses/linear-classifiers-in-python}{Linear Classifiers in Python (4h)}: Logistic regression, SVM, Loss functions
\end{itemize}

\textbf{Data Science with Python (Advanced) - Exchange Pool}

\begin{itemize}
\tightlist
\item
  \href{https://www.datacamp.com/projects/684}{TV, Halftime Shows and the Big Game (4h)}
\item
  \href{https://www.datacamp.com/courses/interactive-data-visualization-with-bokeh}{Interactive Data Visualization with Bokeh (4h)}
\item
  \href{https://www.datacamp.com/courses/introduction-to-time-series-analysis-in-python}{Time Series Analysis (4h)}
\item
  \href{https://www.datacamp.com/courses/machine-learning-for-time-series-data-in-python}{Machine Learning for Time Series Data in Python (4h)}
\item
  \href{https://www.datacamp.com/courses/advanced-deep-learning-with-keras-in-python}{Advanced Deep Learning with Keras (4h)}
\item
  \href{https://www.datacamp.com/courses/data-visualization-with-seaborn}{Data Visualization with Seaborn (4h)}
\item
  \href{https://www.datacamp.com/courses/web-scraping-with-python}{Web Scraping in Python (4h)}
\item
  \href{https://www.datacamp.com/courses/writing-efficient-python-code}{Writing Efficient Python Code (4h)}
\item
  \href{https://www.datacamp.com/courses/unsupervised-learning-in-python}{Unsupervised Learning in Python (4h)}
\item
  \href{https://www.datacamp.com/courses/optimizing-python-code-with-pandas}{Writing Efficient Code with pandas (4h)}
\item
  \href{https://www.datacamp.com/courses/deep-learning-in-python}{Introduction to Deep Learning in Python (4h)}
\item
  \href{https://www.datacamp.com/courses/forecasting-using-arima-models-in-python}{ARIMA Models in Python (4h)}
\end{itemize}

\end{tipsp}

\hypertarget{links-1}{%
\subsection{Links}\label{links-1}}

Official Tutorials/Documentation:

\begin{itemize}
\tightlist
\item
  \url{https://docs.python.org/3/tutorial/index.html}
\item
  \url{https://jupyter.org/documentation}
\end{itemize}

Further Explanations:

\begin{itemize}
\tightlist
\item
  \url{https://pythonprogramming.net/}
\item
  \url{https://automatetheboringstuff.com/}
\item
  \url{https://www.reddit.com/r/learnpython}
\item
  \url{https://www.datacamp.com/community/tutorials/tutorial-jupyter-notebook}
\end{itemize}

\hypertarget{your-data-science-project}{%
\section{Your Data Science Project}\label{your-data-science-project}}

\hypertarget{coding-meetups-and-requirements}{%
\subsection{Coding Meetups and Requirements}\label{coding-meetups-and-requirements}}

Now that you have learned the theoretical foundation in the DataCamp courses, you can put your skills into practice. We have put together a project for you based on real data sets. You can read about the details in the following chapters of this project guide.\\
Of course, we will also go into more detail about the project and the tools that go with it. We will discuss everything you need to know during the first Coding Meetup, which will take place on \textbf{December 9, 2020}. After that, the work on the project will officially begin.
You can find the exact project tasks together with further explanations and hints in the following chapters.

To receive the certificate, it is essential that you have solved at least two thirds of the ``Exploratory Data Analysis'' part of the project. For the advanced participants, the entire ``Price Prediction -- The Application of Statistical Models'' part is added. In addition, two thirds of the respective curriculum on DataCamp must be completed. You can find more detailed information on this in the ``Curriculum'' section of the respective programming language above.

\newpage

\hypertarget{introduction-to-your-project}{%
\chapter{Introduction to Your Project}\label{introduction-to-your-project}}

\hypertarget{purpose-of-the-project-guide}{%
\section{Purpose of the Project Guide}\label{purpose-of-the-project-guide}}

Welcome to the project guide for your TechAcademy Data Science project! This document will guide you through the different steps of your project and will provide you with useful hints along the way. However, it is not a detailed step by step manual, because we felt like it was important that you develop the skills of coming up with your own way of solving different tasks. This is a great way to apply the knowledge and tools you have acquired through DataCamp.

It might happen that questions come up or that you don't know how to solve a task right away---but don't worry---this is just part of coding. In those cases you can also find helpful links in the introductory chapters, where your questions might already have been answered. If not, and in the unlikely case that even Google can't help you, the TechAcademy mentors will help you via Slack or directly during the coding meetups.

\textbf{At the end of the project guide you will find an overview of all tasks that have to be completed}, depending on your track (beginner/advanced). You can use this list to check which tasks still need to be completed and which tasks are relevant for your track.

\hypertarget{what-is-this-project-about}{%
\section{What is this Project About?}\label{what-is-this-project-about}}

Last semester, we dealt with the COVID-19 outbreak, but to take your mind off things, we thought it would be a good idea to focus on a different topic this semester: the business model of Airbnb. More precisely, we are analyzing part of a very detailed data set of all Airbnb offers in Berlin. The data was scraped in November 2018, which means it was taken directly from the Airbnb website itself. You will find all kinds of information in the data set - useful and useless ones.

Are you already curious to see for yourself? In analogy to the typical Data Science workflow, we split this project into two parts. First you are going to learn how to perform an Exploratory Data Analysis (EDA). You will have a closer look at the data, transform it and then get to know the different variables and what they look like in different types of visualizations. Beginners will have completed the project after this, but it will be beneficial to also try and work on the next part: In the second part of the project you will come up with a model that predicts Airbnb prices in Berlin as accurately as possible. You are going to start with a linear regression model, which you can modify as you please and then you can explore all the other possibilities of modeling and predicting data.

But first things first: What exactly is EDA and what can you achieve with it?

\hypertarget{exploratory-data-analysis-getting-to-know-the-data-set}{%
\section{Exploratory Data Analysis -- getting to know the data set}\label{exploratory-data-analysis-getting-to-know-the-data-set}}

As a first step you will get to know the data set. This means you will describe the data and answer questions like ``Which variables are contained in the data set? \ldots and how are they related?''. For this you can often use graphical tools like box plots or histograms.

This first part of the project is structured in a way that lets you get to know the data thoroughly by completing the given tasks one after the other. As a beginner, you can stop after this part, because you will have fulfilled the necessary coding requirements for the certificate. However, if this first part inspires you to learn more, we encourage you to also work on the second part.

This project guide is structured in the following format. Since the concept of Data Science is independent of specific programming languages, we will describe the general approach in this part of the text. After you understood the overall concept and the tasks we are asking you to do, you will find language-specific tips and tricks in visually separated boxes. If you participate in our \texttt{R}-program, you'll only need to look at the boxes with the blue border. Conversely, you only need to look at the yellow-bordered boxes if you are coding in \texttt{Python}. From time to time it might be interesting to check out the other language -- though you can do the same in both, they sometimes have a different approach to the identical problem. It makes sense that you complete the first few beginner chapters mentioned in the introductory chapter. We recommend that you finish the courses at least until and including \emph{Exploratory Data Analysis} for both tracks.

\hypertarget{prediction-apply-statistical-methods}{%
\section{Prediction -- Apply Statistical Methods}\label{prediction-apply-statistical-methods}}

This part is mainly for the advanced TechAcademy participants. If you are a beginner and you were able to complete the first part without too many difficulties, we highly recommend trying to do the second part as well. Statistical models are a major part of data science and this is your chance of developing skills in this area.

You got to know the data in the first part and you should be familiar with it so that it is now possible to use it to make predictions about Airbnb prices based on information you have about the apartments. After having completed the second part, you will send us your predictions and we will then check how accurate your model was. The best model will win!

For this part of the project, we recommend the advanced courses mentioned in the introductory chapter. Please note that there are more courses available so if you want to extend your skills even further, feel free and complete more courses on the topics that interest you. We recommend that you finish the courses at least until and including \emph{Unsupervised Learning in Python} for the \texttt{Python} track and \emph{Machine Learning Toolbox} for the \texttt{R} track.

Ready? After getting a first impression of what this project is all about, let's get started!

\newpage

\hypertarget{exploratory-data-analysis}{%
\chapter{Exploratory Data Analysis}\label{exploratory-data-analysis}}

At the end of this chapter, you will look into your personal Spotify streaming history. You need to request this data from Spotify, and this can take a few days. Therefore, you should ask for your data as early as possible. Then, start with the first exercises of this chapter which will use some of the data we provide.

\textbf{Download your personal Spotify data}

If you do not use Spotify personally, you can skip this step and start with chapter 4.1.
In chapter 4.5, we will use the requested personal Spotify data, and you can use the data from one of your team members. If nobody on your team uses Spotify, we will provide you with some dummy data.

We will follow eight steps based on \href{https://www.makeuseof.com/tag/download-privacy-data-spotify/}{this} blog post by Dan Price to download your personal Spotify data:
1. Go to \url{https://accounts.spotify.com/en/status} and enter your login credentials.
2. In the menu on the screen's left-hand side, click on Privacy settings.
3. Scroll down to the Download your data section. You will see the following:

\begin{center}\includegraphics[width=1\linewidth]{splots/spotify} \end{center}

\begin{enumerate}
\def\labelenumi{\arabic{enumi}.}
\setcounter{enumi}{3}
\tightlist
\item
  Click on \emph{Request}. You will probably have to complete a captcha and will receive a confirmation email.
\item
  Click ``Confirm'' in the email. This confirmation will take you back to Spotify, where you will receive an on-screen notification saying:``We are preparing your data file. This {[}process{]} can take up to 30 days to complete. You'll receive an email when it is ready to download.'' - Spotify
\item
  When the data preparation is complete, return to Privacy settings; you will now have access to Step 3.
\item
  Click on Download.
\end{enumerate}

In the next couple of days, you will receive an email from Spotify to download your personal Spotify data as a \texttt{my\_spotify\_data.zip} file. Unzip the file and look for the \texttt{StreamingHistory0.json} file.

\textbf{Hint}: Check what Spotify knows about you by inspecting \texttt{Userdata.json}, \texttt{SearchQueries.json}, or \texttt{Inferences.json} files using Notepad (Windows), TextEdit (MacOS), or VIM (Linux). Surprised? :)

\hypertarget{getting-started}{%
\section{Getting Started}\label{getting-started}}

In this chapter, you will apply the knowledge that you gained throughout the semester to real-world data. The exercises will cover typical problems that you usually encounter when analyzing data.

Before you can dive into the data, set up your programming environment. This will be the place where the magic happens -- all your coding will take place there.

\begin{tips}r

In your workspace on \href{https://rstudio.cloud/projects}{rstudio.cloud}, we have already uploaded an ``assignment'' for you (Template Airbnb). When you create a new project within the workspace \emph{Class of '21 \textbar{} TechAcademy \textbar{} Data Science with R}, your workspace will open up. We've already made some arrangements for you: The data sets you will be working with throughout the project are already available in your working directory. We also created an `RMarkdown' file, with which you will be able to create a detailed report of your project. You can convert this file into a PDF document when you have completed the documentation. Open the file ``Markdown\_Spotify.Rmd'' and see for yourself!

\end{tips}

\begin{tipsp}p

We recommend using \href{https://colab.research.google.com}{Google Colab} for this project since it requires no particular setup, stores its notebooks to your Google Drive, and makes it easy for you to share them with your team members.

As an alternative to Google Colab, you might want to install Jupyter Notebook locally using the Anaconda distribution.
Either way, when importing Airbnb data, you can use the links to the respective data files provided in the ``Data-Links'' document, which you will find in the TechAcademy drive.

We will give you a more detailed step by step demo during the first coding meetup.

\end{tipsp}

\hypertarget{importing-data}{%
\subsection{Importing Data}\label{importing-data}}

First things first, you will need to load the following two CSV files into your coding environment:

\begin{itemize}
\tightlist
\item
  \href{https://drive.google.com/uc?export=download\&id=1OmFrdUaPFatpPLRikLJHS-VoF1F27G73}{spotify\_songs.csv}
\item
  \href{https://drive.google.com/uc?export=download\&id=1gHO566_qxBbz7t8fcH10uVQDAmxK33IF}{song\_features.csv}
\end{itemize}

How is the data structured, and which variables does it contain?To get a peek at the data you'll be working with, output the first/last few rows.Also, compute some basic descriptive statistics (count, mean, and standard deviation) and determine the column's data types.

\begin{tips}r

First, the data set has to be loaded from your folder structure into the workspace. Import both the \texttt{song\_features.csv} and the \texttt{spotify\_songs.csv} data sets into your workspace (e.g., using \{base\} R's \texttt{read.csv()}, \href{https://readr.tidyverse.org}{\{readr\}}'s \texttt{read\_csv()} or \href{https://rdatatable.gitlab.io/data.table/}{\{data.table\}}'s \texttt{fread()} function) and name the object(s) accordingly.

Though you can name the object however you wish, we recommend using \texttt{original\_file\_name} + e.g., \texttt{\_df} (for data.frame) / \texttt{\_tbl}(for tibble) / \texttt{\_DT} (for data.table) naming convention, so that you can easily distinguish your R objects (data frames, plots, vectors, etc.) later on.

Now, get an overview of the data set. Are there any variables you could drop while reading the data set? You can use the following functions to print the data frame, for example \texttt{head()} or \texttt{glimpse()}.

Lastly, get a quick summary of the data using \{base\} R's \texttt{summary()} or \href{https://personality-project.org/r/psych/}{\{psych\}}'s \texttt{describe()} function.
To find more ways to generate quick data summaries in R, check \href{https://dabblingwithdata.wordpress.com/2018/01/02/my-favourite-r-package-for-summarising-data/}{this} blog post from Adam Medcalf.

\end{tips}

\begin{tipsp}p

Start by importing the pandas' library: \texttt{import\ pandas\ as\ pd}.
You can use the same panda's method you used to import CSV data from your local machine to import CSV data from a web server: Just replace the file path with the URL of the CSV file.
The URLs are:
spotify\_songs.csv: \url{https://tinyurl.com/TechAcademy-songs}
song\_features.csv: \url{https://tinyurl.com/TechAcadey-song-features}
Finally, get familiar with the data by outputting some of the data frame (e.g.~\texttt{df.head()}), as well as some basic information such as column data types and some descriptive statistics (\texttt{df.info()} \texttt{df.describe()}).

\end{tipsp}

\hypertarget{convert-milliseconds-to-minutes}{%
\subsection{Convert Milliseconds to Minutes}\label{convert-milliseconds-to-minutes}}

\textbf{We will start with the data set ``spotify\_songs.csv''.}

You might have noticed a column with song durations given in milliseconds from the task before. Since (presumably) no one measures song durations in milliseconds, convert them into minutes.

\begin{tips}r

You could use \href{https://dplyr.tidyverse.org/index.html}{\{dplyr\}}'s \texttt{mutate} function to compute and add a new column \texttt{duration\_min} using the existing column \texttt{duration\_ms}. If you haven't heard of the \{dplyr\} package yet, take the respective \href{https://www.datacamp.com/courses/data-manipulation-with-dplyr}{DataCamp course} and the DataCamp's course on \href{https://www.datacamp.com/courses/exploratory-data-analysis}{Exploratory Data Analysis}. Use the respective function and add the new variable to your existing data set.

You can use \href{https://www.timecalculator.net/milliseconds-to-minutes}{this} online calculator as a point of reference for the needed \texttt{mutate()} calculation.

\end{tips}

\begin{tipsp}p

Compute the length of songs in minutes based on the column ``duration\_ms'' and assign it to a new column with

\texttt{df{[}“duration\_minutes”{]}=\ ...}

\end{tipsp}

\hypertarget{song-durations}{%
\subsection{Song Durations}\label{song-durations}}

Next, we want to focus on the concept of distribution:
In this task, you should focus on computing the mean and standard deviation of the duration in minutes and finding the song with the longest duration.
Ideally, the output should be well formatted to something like ``The song with the longest duration is \texttt{{[}song{]}\ by\ {[}artist{]}}.''

\begin{tips}r

Simply use \{base\} R's \texttt{mean()} function to compute a mean of any column in R. Similarly, R comes with \texttt{sd()} function to calculate the standard deviation of a column. You can combine these two functions, e.g., in a \{dplyr\}'s \texttt{summarise()} verb.

Next, find the song with the longest duration (in minutes) using the newly created \texttt{duration\_min} column. The \texttt{filter()} verb from \{dplyr\} in combination with \texttt{max()} function could be helpful in this case. Moreover, you can combine \{dplyr\}'s \texttt{select()} verb with the mentioned \texttt{filter()} verb to show: song's duration, its artist(s) and its name.

For example:

\begin{longtable}[]{@{}rrr@{}}
\toprule
duration\_min & artist(s) & song name \\
\midrule
\endhead
X.X & YY & ZZ \\
\bottomrule
\end{longtable}

\end{tips}

\begin{tipsp}p

Calculating mean and standard deviation should be reasonably easy by applying the mean() and std() function to the corresponding column.

To find the song with the longest duration, you can use pandas \texttt{nlargest} function. You can apply it to a data frame df with \texttt{df.nlargest(n,\ column)}, where n is the number of rows to keep and \texttt{column} defines in which column you want to find the largest values.

\end{tipsp}

\hypertarget{visualizing-song-distributions}{%
\subsection{Visualizing Song Distributions}\label{visualizing-song-distributions}}

Visualizing data is essential to facilitate perception and understanding of information: Create a graph to visualize the distribution of song lengths.

We focus on creating a histogram in the tips section since it's perhaps the most common approach for plotting distributions. You can, however, choose a different chart type; just make sure that the information you want to display is clear and correct.
However, as in school, always add axis labels when possible.

\begin{tips}r

You can use \{base\} R's \texttt{hist()} function to generate a simple histogram of song duration (in minutes). However, we encourage you to use the more extensive \href{https://ggplot2.tidyverse.org}{\{ggplot2\}} package. \{ggplot2\}'s syntax is perhaps a bit harder to learn at the start, but it gives you more plotting benefits over the long run. If you haven't heard of the \{ggplot2\} package yet, take the respective \href{https://www.datacamp.com/courses/data-visualization-with-ggplot2-1}{DataCamp course}.

You could, e.g.~install \href{https://github.com/dreamRs/esquisse\#esquisse-}{\{esquisse\}} package to learn the \{ggplot2\}'s syntax easier. \{esquisse\} will provide you a graphical user interface (GUI) inside RStudio to generate \{ggplot2\} figures. You can then copy/paste the syntax that generated the \{ggplot2\} figure to your R Markdown file. Awesome, eh?

\end{tips}

\begin{tipsp}p

With matplotlibs' \texttt{pyplot.hist(...)}, you can plot a histogram in one line of code.

With some more lines of code, you could (and should) add axis labels.
Try adding a title to the histogram, change the color to your preferences, etc.
Maybe you could even delete outliers with methods such as \texttt{col1.between()} and \texttt{col1.quantile()} - but don't bother with details early on and leave them to the end in case you have enough time.

\end{tipsp}

\begin{center}\includegraphics[width=1\linewidth]{splots/durations (3)} \end{center}

\hypertarget{time-series-data}{%
\section{Time Series Data}\label{time-series-data}}

\hypertarget{visualizing-song-durations-over-time}{%
\subsection{Visualizing Song Durations over Time}\label{visualizing-song-durations-over-time}}

While at the topic of song durations, it might also be interesting to see how the average song length evolved: Choose and plot the chart you think is most appropriate for displaying this type of information.

After plotting, please comment/interpret the graph: Were there any significant increases/decreases in song length over time? If so, what could be the reason?

\begin{tips}r

You first need to prepare the data before visualizing it. Generate the average duration of a song (in minutes) by each year. You already used \texttt{summarise()} and \texttt{mean()} functions before. Now, you need to generate that same statistic for each year. You can, e.g., use \{dplyr\}'s \texttt{group\_by()} verb for this purpose.

Afterward, use \{ggplot2\} or \{base\} R's \texttt{plot()} to create a chart.

Please answer open questions in RMarkdown outside the code chunk.

\end{tips}

\begin{tipsp}p

There is a pandas method to group your data frame by year. In combination with the \texttt{.mean()} method, you can compute the average song length by year in one line of code.

Plot the result with matplotlib.

Please answer open questions in text cells.

\end{tipsp}

\begin{center}\includegraphics[width=1\linewidth]{splots/durations_over_time} \end{center}

\hypertarget{merge}{%
\subsection{Merge}\label{merge}}

As a data scientist, you'll often find yourself working with data sets from different data sources referencing the same object. For example, you might have the song names in one file and the respective song features in a separate file. It would make more sense to just merge the two data sets into one.

Indeed, this is what happened with our data. So, to make life easier for the upcoming tasks, you'll now need to merge both data sets.
As you might have noticed, there is one column that both data sets share. Find out which column that is and combine the data sets on that column.

\begin{tips}r

You could, e.g., use \{dplyr\}'s \texttt{left\_join()} to combine the two data sets (\texttt{song\_features.csv} and \texttt{songs.csv}) side-by-side.

Take a look at the \href{https://r4ds.had.co.nz/relational-data.html\#understanding-joins}{``R for Data Science''} book for an illustration of join types.

\end{tips}

\begin{tipsp}p

We want to combine the two data sets \texttt{spotify\_songs.csv} and \texttt{song\_features.csv}.
The official documentation of the Pandas library provides a great guide on how to deal with this stuff:
\url{https://pandas.pydata.org/docs/user_guide/merging.html}

\end{tipsp}

\hypertarget{line-plot}{%
\subsection{Line Plot}\label{line-plot}}

Your goal for this task is to plot how `acousticness', `danceability', `energy', `valence', `speechiness', and `instrumentalness' changed over time.

\begin{tips}r

This task involves three steps: creating a summarized data set, reshaping that data set from wide to long format if you'd like to use \{ggplot2\} and finally -- plotting it.

Use the familiar \texttt{group\_by()} and \texttt{summarise()} \{dplyr\} verbs for the first part of the task. Unlike before, include multiple columns in your data summary. Use the familiar \texttt{mean()} function on each column to summarize the data. Save the resulting data summary as a new object.

Use the \href{https://tidyr.tidyverse.org}{\{tidyr\}}'s \texttt{pivot\_longer()} for the second part of the task. \texttt{pivot\_longer()} allows you to transform derived summarized data set from wide to long format by decreasing the number of columns and increasing the number of rows (a.k.a., ``lengthening'' the data). Such transformation is useful for plotting, e.g., in \{ggplot2\}.

Use the \{ggplot2\} to create a line chart of six variables for this task's final step. Alternatively, use the \{base\} R's \texttt{plot()}.

\end{tips}

\begin{tipsp}p

Once again, group the entries by year, get the mean values for the respective column for all years and use matplotlibs' pyplot for plotting.
Don't forget to add a legend and axis labels to the plot.

\end{tipsp}

\begin{center}\includegraphics[width=1\linewidth]{splots/features_over_time} \end{center}

\hypertarget{radar-plot-advanced}{%
\subsection{Radar Plot (Advanced)}\label{radar-plot-advanced}}

\textbf{If you are on an advanced track, do this exercise; otherwise, you can skip it.}

Let's get a little fancy by displaying the yearly average song feature values through an interactive radar chart (aka spider plot or polar plot). This chart is very favorable in some cases when you try to display more than two dimensions. Here, each column (or dimension) gets its axis, and the dots on each axis are connected, resulting in a polygon shape.

The features you'll need to plot are: `acousticness', `danceability', `energy', `valence', `speechiness', and `instrumentalness'.

\begin{tips}r

Use the \href{https://plotly.com/r/getting-started/}{\{plotly\}} package to create an interactive Radar/Spider chart of six columns over time. Similar to task 2.3 before, this task will involve summarizing, transforming, and plotting the data.

You are now already experienced in summarizing the data with, e.g., \texttt{group\_by()} and \texttt{summarise()} \{dplyr\} verbs. Focus on summarizing the six columns.

Afterward, transform the data from wide to long format. You could, e.g., use \{tidyr\}'s \texttt{pivot\_longer()} again or \{data.table\}'s \texttt{melt()}.

Lastly, plot the figure using \{plotly\}. Use \href{https://plotly.com/r/radar-chart/}{this} resource as a point of reference for a static plot. Then, create an interactive animation using \href{https://plotly.com/r/animations/}{this} resource.

\end{tips}

\begin{tipsp}p

Here we work with the visualization library plotly, which is a more sophisticated plotting tool than matplotlib:
\texttt{import\ plotly.express\ as\ px}.

Use the already computed data frame with average feature values for each year. Note that after you grouped the data frame, the \texttt{year} column automatically has been set to be the index column -- reverse that (i.e., ``reset'' the index).
Then apply the \texttt{.melt("year")} method on the data frame and have a look at the result. Before continuing, briefly try to understand what happened.

Done that, you can create the radial plot with the \texttt{px.line\_polar(...)} method. The documentation (google something like ``plotly express radar chart'' to find the documentation) tells you which arguments you need (and can) pass to this method:
- Of course, you need to pass some value for the \texttt{data\_frame} argument. For our purpose, this will be the melted data frame.
- You'll also need to pass values for the keyword arguments \texttt{r}, \texttt{theta}, \texttt{animation\_frame}, \texttt{line\_close}, and \texttt{range\_r}.
The values for these arguments are the strings \texttt{“value”}, \texttt{“variable”}, \texttt{“year”}, the boolean value \texttt{True}, and the tuple \texttt{(0,1)} respectively.

Apply \texttt{.show()} directly after the \texttt{line\_polar(...)} method to display the plot.

As always, try updating font size, layout, etc., as you like.

\end{tipsp}

\begin{center}\includegraphics[width=1\linewidth]{splots/features_over_time_advanced} \end{center}

\hypertarget{artists}{%
\section{Artists}\label{artists}}

\hypertarget{most-productive-artists}{%
\subsection{Most Productive Artists}\label{most-productive-artists}}

Our subsequent interest lies in analyzing music artists: Start by showing which artists wrote the most songs using a horizontal bar plot (or any other plot you think works best).

\begin{tips}r

You could use the familiar \texttt{group\_by()} \{dplyr\} verb for this task. However, it would help if you counted the songs of each artist. Does that ring a bell for another \{dplyr\} verb?

After grouping and counting, sort the result by its count. Then, subset the 15 artists with most songs. You can save the derived data summary to a new object.

Lastly, plot the result using \{base\} R or \{ggplot2\} by creating a bar chart.

\end{tips}

\begin{tipsp}p

Pandas' \texttt{.value\_counts(...)} method is all you need to prepare the data.
For plotting use matplotlib's \texttt{plt.barh(...)}.

\end{tipsp}

\begin{center}\includegraphics[width=1\linewidth]{splots/artists_bar_chart} \end{center}

Congratulations -- based on your work with basic data transformations and many visualizations, you now have a solid understanding of the Airbnb offers in Berlin. With this, you have now successfully completed the first part of the project! If you are in the beginner group, your minimum requirements are hereby met. Nevertheless, we strongly recommend that you also have a look at the following part. There you'll be developing methods to predict the price of an apartment! Sounds exciting, doesn't it?

\newpage

\hypertarget{price-prediction-application-of-statistical-methods}{%
\chapter{Price Prediction -- Application of Statistical Methods}\label{price-prediction-application-of-statistical-methods}}

In the previous part you got a feel for the data set. You now know which variables are included and a few characteristic properties of the data set. But so far, we have only visualized the data set. In this section we go a step further and use statistical methods to predict the price of individual Airbnb apartments as accurately as possible.

In order to be able to compare your models at the end, we use a uniform metric according to which we can check the price predictions for accuracy. In our case this is the Root Mean Square Error (\(RMSE\)), i.e.~the root of the average squared difference between the predicted (\(\hat{y}_i\)) and actual value (\(y_i\)):

\[ RMSE = \sqrt{\frac{1}{N}\sum_{i=1}^{N}{(\hat{y}_i-y_i)^2}} \]

The closer the \(RMSE\) is to 0, the better your model predicts prices. In the following, your goal is therefore to reduce the \(RMSE\) of your various models as much as possible through continuous improvements.

We use three different data sets for the following part, which you can find in the subfolder \emph{Full Data Set}. These are based on a much more extensive data set that contains a total of 96 variables for each apartment. We have already done the test / train / validation split of the data so that each group works with the same data sets.

Here is a brief description of what you need each of the data sets for:

\begin{itemize}
\item
  \emph{train.csv} (60 \%): You use this training data set to train your model. The model can learn the relationships between the variables based on the training data set that contains both, the variables needed to predict the prices and also the actual prices themselves.
\item
  \emph{test.csv} (30 \%): With this test data set you can test how well your model predicts the price using data that has not been seen before. This will help you for example with recognizing under- or overfitting.
\item
  \emph{val.csv} (10 \%): We have removed the \texttt{price} variable in this validation data set. At the end you apply your best model to it and send us your predictions for the variable \texttt{price}.
\end{itemize}

We then compare these with the actual values (only known to us) with the help of the \(RMSE\) and will then choose the best model across all groups.

Note that some adjustments are necessary in these three data sets. For example, all price variables are marked with a \$ sign. As before, we need to remove these in order to be able to transform the values to a numeric format. Always keep in mind that you have to apply all transformations to all three data sets, otherwise you will not be able to apply your trained model to the test and validation data sets.

\hypertarget{examine-the-correlation-between-the-variables-train}{%
\section{Examine the Correlation Between the Variables (train)}\label{examine-the-correlation-between-the-variables-train}}

First load the \texttt{train.csv} data set from the \emph{Full Data Set} folder into your workspace. Now look at the variable names and the first entries to decide which variables could be useful for predicting the price. Select these (limit yourself to no more than 20 variables at first) plus \texttt{price} and save it in a new data frame.

How are the individual variables related to each other? In other words, to what extent do the variables in the data set correlate with one another? Finding this out is extremely important for deciding which model specification to use later. A good place to start is to create a correlation matrix:

\begin{center}\includegraphics[width=1\linewidth]{plot/5_1_ggpairs} \end{center}

\begin{tips}r

Part of that is function \texttt{cor()} from the base package. Select all numerical variables in your data set with the help of \texttt{sapply()} and create a correlation matrix.

The \texttt{GGally} package with the \texttt{ggpairs()} function provides a very practical plot for visualizing relationships between many variables.

Select the four variables (and \texttt{price}) that you think most influence the price and create a \texttt{ggpairs} plot. Note that the plot quickly becomes illegible and takes a long time to create as soon as you plot significantly more than five variables.

\end{tips}

\begin{tipsp}p

A handy library for plotting correlation matrices is the seaborn library: \texttt{import\ seaborn\ as\ sns}.

Use its pairplot method and pass on the dataframe with the selected columns to visualize distributions and correlations.

Additionally, you may want to plot a heatmap with \texttt{sns.heatmap()} which makes it even easier to see correlations.

\end{tipsp}

Which of your examined variables correlates most with price and which seems to be more independent from price? You now have a first impression over which variables could be important for your model. So let's get to your first price prediction model!

\hypertarget{first-predictions-with-simple-regression-models-train}{%
\section{First Predictions with Simple Regression Models (train)}\label{first-predictions-with-simple-regression-models-train}}

Now you can make use of your statistical knowledge. You will need a method with which you can predict the price of an Airbnb apartment for a specific day. A very simple first approach would be to use the average demand as the first prediction. However, this is almost certainly not the best prediction. In this case, your predicted price would be the same over all days and would ignore all factors that influence the price.

Ever heard of linear regression? That would be a much better approach. Now you can use your statistics skills. First set up a model with the dependent variable \texttt{price}. In the previous exercise you examined different variables. Now choose the variable with the highest correlation to price and use that as the only independent variable.

For example, your first regression model could look like this:

\[price_i = \beta_0 + \beta_1 bedrooms_i + \epsilon_i\]

\begin{tips}r

In \texttt{R} you can implement a simple linear regression with the function \texttt{lm()}. You then get the model summary with the \texttt{summary()} function.

\end{tips}

\begin{tipsp}p

Define both dependent (y\_train) and independent (X\_train) variables and clean the data if necessary.

For the next step the X\_train values need to be reshaped \texttt{.values.reshape(-1,1)}.
Note: If you use more then one feature you don't have to reshape your data.

Import \texttt{LinearRegression()} from \texttt{sklear.linear\_model} and train your model using \texttt{LinearRegression().fit()}.

\end{tipsp}

Does your independent variable have a statistically significant impact on apartment price? Probably yes, because we selected the variable most correlated with price. However, if we stick to this very simplified model, we are making a typical mistake: the so-called Omitted Variable Bias (OVB). To put it simply, we neglect (in statistics jargon: ``do not control for'') variables that have a significant influence on the dependent variable. One could suspect that other influencing factors also play a large role in price setting. If we do not include them, the estimate of the effect of \texttt{bedrooms}is biased and thus hardly useful. In our case this is not a big problem for the time being, since we are not interested in causal effects, but rather in the best possible prediction. Your statistics professor would almost certainly object to such a model. Nonetheless, with just a single explanatory variable, this model will not necessarily predict the price well.

One possible solution is to simply include the omitted variables in the model -- how practical that these are already included in the data set. So let's set up a somewhat more extensive model that includes one more variable:

\[price_i = \beta_0 + \beta_1 bedrooms_i + \beta_2 cleaning\_fee_i + \epsilon_i\]

Now compare the results of the two models. Does the larger model explain a higher proportion of the variance in price? In other words, which model shows the higher value for the \(R^2\) measure?

\begin{tips}r

You can easily include such LaTeX tables in your RMarkdown document with the \texttt{stargazer} package:

\begin{center}\includegraphics[width=1\linewidth]{plot/5_table} \end{center}

\end{tips}

\hypertarget{from-training-to-testing-making-predictions}{%
\section{From Training to Testing -- Making Predictions}\label{from-training-to-testing-making-predictions}}

You have now trained your first model with the training data set. But how well does the model handle data that it has not seen yet? This is a very important test to evaluate the quality of your model.

Has your model only ``memorized'' the existing patterns in the training data set?
Then the relationships from the training data set would not be transferable to the test data set. With so-called overfitting, the model was trained too closely to the training data set and therefore provides poor predictions for unknown data -- for example the data in your test and validation data sets.

On the other hand, there is also the problem of underfitting: Your model has not sufficiently learned the actual relationships between the data and therefore makes poor predictions in the test data set. So it is important to find a balance between the two problems.

Now the distinction between training and test data sets becomes important. As a reminder: we use train data to train a model and the test data to ultimately test the quality of our model.

Now load the data set \texttt{test} in addition to the data set \texttt{train} that you have already used. In order to test your model on previously unseen data, you can apply the model to the test data set.

\begin{tips}r

Use the \texttt{predict} function for this:

\texttt{predicted\_price\ \textless{}-\ predict(your\_saved\_model,\ your\_test\_data\_frame)}

You have now created a vector with all price predictions for the test data set. You can now compare this with the actual values for \texttt{price} from \texttt{test}.

In order to use a uniform comparison metric, please use the following function to measure your prediction accuracy:

\begin{verbatim}
# Function that returns Root Mean Squared Error while ignoring NAs
rmse <- function(actual, predicted) {
sqrt(mean((predicted - actual)^2, na.rm = TRUE))
}
\end{verbatim}

\end{tips}

\begin{tipsp}p

After training, import the data from the \texttt{test.csv} file, define both variables \texttt{X\_test}and \texttt{y\_test}, and create a vector with price predictions applying \texttt{.predict(X\_test)} on your model. Store your prediction in the variable \texttt{predicted\_price}.

Finally, compare your predicted values with the test values:

\begin{verbatim}
from sklearn.metrics import mean_squared_error
# Function that returns Root Mean Squared Error while ignoring NaNs
rmse = mean_squared_error(y_test, predicted_price, squared=False)
\end{verbatim}

\end{tipsp}

Now compare both regression models. Does the larger model have better prediction accuracy, i.e.~a lower \(RMSE\)? Now you have a benchmark for your more advanced models, which you can beat in the next part.

\hypertarget{apply-advanced-machine-learning-algorithms}{%
\section{Apply Advanced Machine Learning Algorithms}\label{apply-advanced-machine-learning-algorithms}}

Now that you have created and tested an initial prediction using a simple regression model, you can now apply more advanced methods. The goal is still to get the lowest possible \(RMSE\) when applying the model to the test data set. Now look at at least one other algorithm and then see if that gives you a more accurate prediction (expressed as a lower \(RMSE\)). You can find inspiration for this in the advanced DataCamp courses, which are listed at the beginning of the project guide. There are no limits for you -- you can refine the regression using certain methods (e.g.~LASSO) or set up a random forest model or a neural network. It is usually a good idea to briefly recall the functionality of the respective algorithms and to consider whether this methodology makes sense in this case with a continuous prediction variable.

At this point, a disclaimer is appropriate: Our data set has a substantial part of missing observations (\texttt{NA}) for many variables. Some machine learning algorithms require a complete set of data with no missing values, while others can do well with a smaller number. If you get into trouble about the missing values, check whether you can impute the missing values. Which method is best for imputation depends heavily on your prediction algorithm.

You can also get a noticeable gain in predictive power by modifying existing variables or generating new variables from the data set (``feature engineering''). For example, we could imagine that the distance from an apartment to the city center has a significant impact on the price. However, this variable is not
included in our data set. You can write a simple function that uses the two coordinate variables to calculate the distance to the center of Berlin and appends this to the data set as a new variable.

Always compare the \(RMSE\) of your advanced models with each other, as well as with the benchmark regression model from before.

Did you find your best model? Then predict prices with your winning model as above -- but this time on the validation data set \texttt{val.csv}.

Attach a \texttt{.csv} dataset in the following format with only the two variables \texttt{id} and \texttt{predicted\_price} as an attachment to your project submission.

\begin{tips}r

You can do this by adding the two vectors \texttt{id} and \texttt{predicted\_price} together and saving them as a \texttt{.csv} file.

\begin{verbatim}
submission <- cbind(val$id, predicted_price_val)
write.csv(submission, "submission_<<TEAM NAME>>.csv", row.names=FALSE)
\end{verbatim}

\end{tips}

\begin{tipsp}p

You may need to do some mapping operations between the train.csv data and the predictions to get to the data frame with id and predicted prices.

Similarly as with the \texttt{pd.read\_csv(path/to/file.csv)} method to load data, you can also write data with the \texttt{df.to\_csv(file.csv)} method.

\end{tipsp}

Congratulations! You've made it to the end of your TechAcademy Data Science project. After visualizing the data in the first part, you've also set up predictive models in this section. If you stuck around until this part and managed to code the better part of the exercises, you've definitely earned your certificate! We hope you had fun learning Data Science with this data set and you enjoyed it -- at least the parts where you didn't get stuck forever because of some unexplainable coding error. Don't forget to send your project results and the prediction-csv file you just generated to our project submission email address before the deadline.

\newpage

\hypertarget{excercise-checklist}{%
\chapter{Excercise Checklist}\label{excercise-checklist}}

This checklist should help you keeping track of your exercises. Remember that you have to hand in satisfactory solutions to at least two thirds of the exercises. If you're part of the beginner track this refers to two thirds of part A (EDA) only. If you're part of the advanced track, you have to hand in at least two thirds of both individual parts A and B. Hence, you cannot hand in 100 percent of the first part and only 50 percent of the second one. You'll need more than 66\% in each one for a certificate. After all, you're not really that advanced if you only did half of it, right?

\textbf{Part A: Exploratory Data Analysis (Beginners + Advanced)}

\begin{enumerate}
\def\labelenumi{\arabic{enumi}.}
\item
  Visualize the available apartments (calendar dataset):

  \begin{enumerate}
  \def\labelenumii{\alph{enumii}.}
  \item
    Transform t/f to boolean values True/False
  \item
    Aggregate data after dates
  \item
    Plot available apartments over time
  \end{enumerate}
\item
  Visualize the listings dataset

  \begin{enumerate}
  \def\labelenumii{\alph{enumii}.}
  \item
    Clean and transform the \texttt{price} and \texttt{cleaning\_fee} columns to numeric values
  \item
    Compute mean and standard deviation of apartment prices for each neighborhood
  \item
    Visualize the price distribution for the most and least expensive district
  \item
    Visualize if \texttt{cleaning\_fee} was used to hide costs
  \end{enumerate}
\item
  Merging of \texttt{listings} and \texttt{review} data set

  \begin{enumerate}
  \def\labelenumii{\alph{enumii}.}
  \item
    Aggregate reviews by \texttt{ID}
  \item
    Merge \texttt{listings} and aggregated reviews data
  \item
    Filter and plot amount of apartments in each district
  \end{enumerate}
\item
  Visualize apartments on map

  \begin{enumerate}
  \def\labelenumii{\alph{enumii}.}
  \item
    Draw a circle for every available apartment on map
  \item
    Colorize each circle according to its districts
  \item
    Draw a 2D-density plot
  \end{enumerate}
\end{enumerate}

\textbf{Part B: Price Prediction Using Statistical Methods ((motivated) Beginners + Advanced)}

\begin{enumerate}
\def\labelenumi{\arabic{enumi}.}
\item
  Visualize feature correlations in a correlation matrix/heatmap
\item
  Regression

  \begin{enumerate}
  \def\labelenumii{\alph{enumii}.}
  \item
    Simple regression model using one variable
  \item
    Improve your model using more features
  \end{enumerate}
\item
  Test your model

  \begin{enumerate}
  \def\labelenumii{\alph{enumii}.}
  \item
    Use test data set to predict prices of the apartments
  \item
    Compare your predictions with the true values
  \item
    Train a new model using a more advanced method, send us your model and the predictions of your best model
  \end{enumerate}
\end{enumerate}

\newpage

\hypertarget{whats-next-in-your-data-science-career}{%
\chapter{What's Next in Your Data Science Career?}\label{whats-next-in-your-data-science-career}}

At the end of your TechAcademy semester, you've successfully coded your way through a whole Data Science project. You liked what you did and would like bring your skills to the next level? Then this section provides you with many useful resources to deepen your knowledge in Data Science in general or Python and R in particular. The first section is useful for every aspiring Data Scientist, while the two following boxes introduce you to some language-specific resources. If you've come across some other useful materials that we didn't mention here, feel free to contact us -- this list is far from complete!

\hypertarget{data-science-in-general}{%
\section{Data Science in General}\label{data-science-in-general}}

\hypertarget{version-control-with-git}{%
\subsection*{\texorpdfstring{Version Control with \href{https://github.com/}{Git}}{Version Control with Git}}\label{version-control-with-git}}

If you're serious about data science, you will need Git. Better learn it early and start enjoying and appreciating it before it's too late and you're pressured into learning it on the fly! Every project you do should be versioned with Git. Regardless if you're working alone or with a big group of developers. Regardless if you write ten lines of code or a really complex program. With Git, you can keep track of all your changes. It's like a Dropbox/Google Drive for developers, but way better. Pro-Tip: Get free GitHub Pro as a Student with the \href{https://education.github.com/pack}{GitHub Student Developer Pack}. Besides all the perks of GitHub Pro, you'll also free access to many other great tools. See the respective tutorials on how to set up your Git workflow.

\hypertarget{advice-for-non-traditional-data-scientists}{%
\subsection*{\texorpdfstring{Advice for \href{https://blog.shotwell.ca/posts/learning_data_science/}{Non-Traditional Data Scientists}}{Advice for Non-Traditional Data Scientists}}\label{advice-for-non-traditional-data-scientists}}

Important advice from Gordon Shotwell, a former lawyer, on what it takes to have a successful data science career coming from a non-computer science background. Extremely encouraging and helpful read on what you should and shouldn't do to reach that goal.

\hypertarget{learn-from-great-data-scientists-on-kaggle}{%
\subsection*{\texorpdfstring{Learn from Great Data Scientists on \href{https://www.kaggle.com/}{Kaggle}}{Learn from Great Data Scientists on Kaggle}}\label{learn-from-great-data-scientists-on-kaggle}}

Kaggle is a platform that hosts data science challenges. The great thing about it is that you can browse through many clever solutions to tricky machine learning tasks. And of course, you can also join the competition and measure your predictions with others. There are plenty of both Python and R notebooks.

\hypertarget{r}{%
\section{R}\label{r}}

\begin{tips}r

\textbf{Install R and RStudio Locally}

RStudio.Cloud is great for getting started with R without having to worry about installing anything locally. Sooner or later you will have to install everything on your own computer. Here's a \href{https://www.datacamp.com/community/tutorials/installing-R-windows-mac-ubuntu}{DataCamp tutorial} on how to do that.

\textbf{Version Control with Git}

RStudio has a nice interface that lets you enjoy the perks of Git without ever having to touch the command line -- sounds great, does it? Learn how to set up the Git \& R workflow with \href{https://happygitwithr.com/}{Happy Git with R}.

\textbf{\href{https://www.r-graph-gallery.com/}{R Graph Gallery}}

Get inspiration to take your plotting to the next level. Includes code to reproduce the plots.

\textbf{Follow the R Master Himself and the R Community}

Hadley Wickham was and continues to be extremely influential on the development of R and its rise to one of the most popular data science languages. He's behind many tools that we taught you in this semester, especially the tidyverse (including great packages such as ggplot2 and dplyr). Follow him \href{https://twitter.com/hadleywickham}{on Twitter} to get great R advice and keep up to speed with everything new to R. Following the many people behind R (not only Hadley) is a great way for acquiring deeper understanding of the language and its developments.

\textbf{Join the \href{https://www.meetup.com/r-frankfurt/}{Campus useR Group in Frankfurt}}

There's a quite active R community in Frankfurt that meets once a month. It's open for students, professors, industry practitioners, journalists, and all people that love to use R. In those meetings, you'll hear about other's work, discuss new developments, and ask questions.

\textbf{Listen to R Podcasts}
Another great way to easily keep up with new developments in the Data Science/R community. Check out
\href{http://nssdeviations.com/}{Not So Standard Deviations} or \href{https://r-podcast.org/}{the R-Podcast}

\end{tips}

\hypertarget{python}{%
\section{Python}\label{python}}

\begin{tipsp}p

\textbf{Install Python Locally}

Until now you've only programmed using JupyterHub on the TechAcademy Server. A next step would be to install Python and Jupyter locally on your computer. This \href{https://docs.anaconda.com/anaconda/install/}{link} contains the necessary information on how to install the software on Windows, iOS or Linux.

\textbf{Choosing the Right Editor}

Using Jupyter is especially useful for short data analyses. But sometimes you want to write longer scripts in Python. In these cases, it is often more convenient to use a code editor instead of Jupyter. \href{https://realpython.com/learning-paths/perfect-your-python-development-setup/}{This tutorial} highlights the positive aspects of such an editor and how to choose the right one for you. Pro Tip: Also check out the other tutorials on \href{https://realpython.com/}{Real Python}.

\textbf{\href{https://python-graph-gallery.com/}{Python Graph Gallery}}

Get inspiration to take your plotting to the next level. Includes code to reproduce the plots.

\textbf{More Advanced Python Concepts}

You know the basic data structures in Python like lists and dictionaries. What are the next steps to improve your knowledge? \href{https://book.pythontips.com/en/latest/index.html}{This website} gives good explanations for slightly more advanced concepts which can be very useful from time to time.

\textbf{A Deeper Understanding}

If you want to get a deeper understanding of the Python programming language and into typical algorithms which are used in the field of Data Science, this \href{https://github.com/ab-anand/py-books/blob/master/Data\%20Science\%20from\%20Scratch-\%20First\%20Principles\%20with\%20Python.pdf}{free book} can be a good starting point.

\textbf{Writing Beautiful Python Code}

``My code doesn't look nice, but it works!''
This might work for yourself, but often you will work on code with other people. But even if you're just coding for yourself it's a good idea to follow the PEP8 style guide. It's a useful convention on how to structure and code in Python. You'll find useful resources for PEP8 \href{https://realpython.com/python-pep8/}{here} and \href{https://www.python.org/dev/peps/pep-0020/\#id2}{here}.

\textbf{Listen to Python Podcasts}

When you don't have time for books you can listen to \href{https://talkpython.fm/home}{Talk Python} or the \href{https://www.pythonpodcast.com}{Python Podcast}.

\end{tipsp}

  \bibliography{book.bib,packages.bib}

\end{document}
